\documentclass[a4paper, 12pt]{article}
\usepackage{biblatex}
\usepackage{amsmath}
\usepackage{amssymb}
\usepackage{braket}
\addbibresource{quantum-latin-squares.bib}
\title{Quantum Latin Squares Overview}
\author{Abdul Fatah Jamro,\\  {Ian Mcloughlin}
}
\begin{document}
\maketitle

\section{Introduction}
Latin squares are the mathematical objects that have apllications in different areas 
like combinatorial designs, scheduling,games such as sudoku puzzels and error correcting codes.
Quantum Latin squares are objects from combinatorics that are a quantum analogue
to Latin squares. \\\textbf{Latin squares} are defined as an n-by-n array of elements of the 
Hilbert space Cn, such that every row and every column is an orthonormal basis. 
They interact with a finite-dimensional commutative C*-algebra by the linear maps
being unitary. They have applications in quantum error correction, quantum key 
distribution, quantum state determination, and quantum pseudotelepathy. ~\cite{Musto2019-vm}


\subsection{Error correcting codes}
Codes can be constructed with various ways.The idea to construct codes with latin squares
is not recent. Binary data is the strings of $0's$ and $1's$. Error correcting codes 
add some extra digits for detection and correction of errors ~\cite{QECC-QLS} 
In the 1960s and 70s Latin squares found use in producing error correcting codes
in classical information theory. We are going to build latin squares' analogy to
quantum information theory in field of quantum error correction (QEC).

\subsection{Latin Square}
A latin square of order $q$ is a $q$ x $q$ array whose entries are from a set of q distinct 
element such that every element is contained exactly once in each row and each column.

\subsection{Orthogonality}
Latin squares A = $(a_{ij})$ and B = $(b_{ij})$ of order $q$ are said to be mutallay orthogonal
if the $q^2$ ordered pairs $(a_{ij}, b_{ij}), i,j = 1,2,3,...,n$ are all distinct. 
\begin{center}
\begin{math}
\begin{bmatrix}
  a & b & c\\
  b & c & a\\
  c & a & b
\end{bmatrix}
\end{math}
\noindent and
\begin{math}
\begin{bmatrix}
  a & b & c\\
  c & a & b\\
  b & c & a
\end{bmatrix}
\end{math} \\
\end{center}
superimposing two orthogonal latin squares will give order pairs as bellow
\begin{center}
$(a,a) (b,b) (c,c)$\\ $(b,c) (c,a) (a,b)$\\ $(c,b) (a,c) (b,a)$ \\
\end{center}
\section{Quantum latin squares, Orthonormal basis}
\newcommand{\C}{\mathbb{C}}
\textbf{definition} A quantum latin square (QLS) of order $n$ is an $n$ -by-$n$ array of elements of the
Hilbert space \( \C^4 \), such that every row and every column is an orthonormal basis.
Bellow is the QLS described in terms of computational basis 
elements $\{ \ket{0}, \ket{1}, \ket{2}, \ket{3} \}$.
\begin{center}
\begin{tabular}{||c|c|c|c||}
  \hline
  $\ket{0}$ & $\ket{1}$ & $\ket{2}$ & $\ket{3}$ \\
  \hline
  $\frac{1}{\sqrt{2}}(\ket{1}-\ket{2})$ & $\frac{1}{\sqrt{5}}(i\ket{0}+2\ket{3})$ & $\frac{1}{\sqrt{5}}(2\ket{0}+i\ket{3})$ & $\frac{1}{\sqrt{2}}(\ket{1}+\ket{2})$\\
  \hline
  $\frac{1}{\sqrt{2}}(\ket{1}+\ket{2})$ & $\frac{1}{\sqrt{5}}(2\ket{0}+i\ket{3})$ & $\frac{1}{\sqrt{5}}(i\ket{0}+2\ket{3})$ & $\frac{1}{\sqrt{2}}(\ket{1}-\ket{2})$\\
  \hline
  $\ket{3}$ & $\ket{2}$ & $\ket{1}$ & $\ket{0}$\\ 
  \hline
\end{tabular}
\end{center}

A classical Latin square of order n is an n-by-n array of integers in the range 
$\{0,...,n-1\}$ such that every row and column contains each number exactly once.
By interpreting a number  
$ k \epsilon \{0,...,n - 1\}$ as a computational basis element $\ket{K_i} \epsilon C_n$,
we can turn an array of numbers into an array of Hilbert space elements:~\cite{musto2016quantum}

\begin{center}
\begin{tabular}{||c|c|c|c||} 
   \hline
   3 & 1 & 0 & 2 \\
   \hline
   1 & 0 & 2 & 3\\
   \hline
   2 & 3 & 1 & 0\\
   \hline
   0 & 2 & 3 & 1\\ 
   \hline
 \end{tabular} $\rightarrow$
\noindent
\begin{tabular}{||c|c|c|c||}
  \hline
  $\ket{3}$ & $\ket{1}$ & $\ket{0}$ & $\ket{2}$ \\
  \hline
  $\ket{1}$ & $\ket{0}$ & $\ket{2}$ & $\ket{3}$\\
  \hline
  $\ket{2}$ & $\ket{3}$ & $\ket{1}$ & $\ket{0}$\\
  \hline
  $\ket{0}$ & $\ket{2}$ & $\ket{3}$ & $\ket{1}$\\ 
  \hline
\end{tabular}
\end{center}
\printbibliography
\end{document}

